\begin{titlepage}
    \begin{center}
        {\huge Title\\}
        \vspace{1.5cm}
        \begin{Large}

            Einsendeaufgabe\\
            Übung: infEthik-01a – Ethik in der Informatik\\
            im Fach Informatik\\
            der Technischen Fakultät\\
            der Christian-Albrechts-Universität zu Kiel\\
            Wintersemester 2023/24\\
        \end{Large}
        \vspace{5.0cm}

        vorgelegt von\\
        Klaas Pelzer\\
        1184754\\
        stu244352@mail.uni-kiel.de\\
    \end{center}

    \vfill
    Prüfer*in:\\
    Kiel, 26.11.2023
\end{titlepage}

\section*{Eigenständigkeitserklärung}
Name, Vorname: Pelzer, Klaas\\
Matrikelnummer: 1184754\\

Hiermit versichere ich, dass ich die Abschlussarbeit im Modul \enquote{Ethik in der Informatik} selbständig verfasst und keine
anderen als die angegebenen Quellen und Hilfsmittel benutzt habe.
Ich habe alle Passagen, die ich aus gedruckten Schriften oder digital verfügbaren Dokumenten übernommenen habe,
gekennzeichnet und korrekt zitiert.
Ferner versichere ich, dass ich die vorliegende Arbeit in keinem anderen Prüfungsverfahren eingereicht habe.
\section{Brauchen wir mehr Richtlinen in der Informatik?}
Mit dem Einzug von neuer KI-Technologien wie selbstfahrenden Autos und KI-Assistenzen nehmen informationalle Systeme immer mehr Einfluss
auf unsere Gesellschaft.
Durch den starken Konkurenzdruck werden Softwareprodukte von den großen IT-Firmen in einem immer steigenderen Tempo weiterentwickelt.
Dabei spielen ethische Fragen meist keine oder eine untergeordnete Rolle, da diese den technologischen Fortschritt aufhalten.
Dies zeigt die kürzliche Entlassung des gesamten \enquote{Ethics and Social Teams} des Unternehmens 
Microsoft\parencite{schifferMicrosoftJustLaid2023}.

Im Essay \citetitle*{vardiAreWeHaving2018} nimmt \Citeauthor*{vardiAreWeHaving2018} Stellung zu der Krise, dass die großen Konzerne aus
dem Silicon Valley ungehindert und ohne Rechenschaft ihre Produkte vertreiben dürfen\parencite{vardiAreWeHaving2018}.
Er stellt die These auf, dass es sich diese Krise nicht mit ethischen Untersuchnugen lösen lässt, sondern dass es neuer Regularien 
und Gesetze bedarf.
Hierfür wird unter anderem der Vergleich zum Autoverkehr gezogen, in dem es ebenfalls erst durch Gesetzte und Normen eine Reduzierung von
Verkehrstoten erreicht werden konnte und nicht durch ethische Belehrungen.
Des Weiteren kritisiert \Citeauthor{vardiAreWeHaving2018} das Geschäftmodell des \enquote{Überwachungskapitalismus} der Internet-Firmen
und nennt in diesem Zusammenhang den Cambridge-Analytica-Vorfall.


 

\newpage
